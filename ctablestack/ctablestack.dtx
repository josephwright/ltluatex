% \iffalse meta-comment
%
% Copyright 2015
% The LaTeX3 Project and any individual authors listed elsewhere
% in this file.
%
% It may be distributed and/or modified under the conditions of
% the LaTeX Project Public License (LPPL), either version 1.3c of
% this license or (at your option) any later version.  The latest
% version of this license is in the file:
%
%   http://www.latex-project.org/lppl.txt
%
%
%
%<package>\ifx
%<package>  \ProvidesPackage\undefined\begingroup\def\ProvidesPackage
%<package>    #1#2[#3]{\endgroup\immediate\write-1{Package: #1 #3}}
%<package>\fi
%<package>\ProvidesPackage{ctablestack}
%<*driver>
\ProvidesFile{ctablestack.dtx}
%</driver>
%<*package>
  [2015/08/01 v1.0a Catcode table stable support]
%</package>
%<*driver>
\documentclass{ltxdoc}
\GetFileInfo{ctablestack.dtx}
\begin{document}
\title{\filename\\Catcode table stable support}
\author{David Carlisle and Joseph Wright}
\date{\filedate}
\maketitle
\setcounter{tocdepth}{2}
\tableofcontents
\DocInput{\filename}
\end{document}
%</driver>
% \fi
%
% \CheckSum{0}
%
% \section{Overview}
%
% This small package adds support for a stack of category code tables to
% the core support for Lua\TeX{} provided by \textsf{ltluatex}. As such, the
% code here may be used with both plain \TeX{} and \LaTeX{}, and requires
% either an up-to-date \LaTeX{} kernel (2016 onward) or the \textsf{ltluatex}
% package.
%
% \noindent
% \DescribeMacro{\setrangecatcode}
% |\setrangecatcode{|\meta{start}|}{|\meta{end}|}{|\meta{catcode}|}|\\
% Sets all characters in the range \meta{start}--\meta{end} inclusive to
% have the \meta{catcode} specified.
%
% \noindent
% \DescribeMacro{\@pushcatcodetable}
% \DescribeMacro{\@popcatcodetable}
% |\@pushcatcodetable|\\
% |\@popcatcodetable|\\
% This pair of commands enable the current category code r\'{e}gime to
% be saved and restored meaning that arbitrary catcode changes can be made.
% This functionality will normally be used in concert with applying
% catcode tables. For example
% \begin{verbatim}
% \catcode`\Z=4 %
% \@pushcatcodetable
% \catcodetable\catcodetable@latex
% % Code here
% \@popcatcodetable
% \showthe\catcode`\Z
% \end{verbatim}
% will ensure that the `content' is set with normal category codes but
% allow restoration of the non-standard codes at the conclusion. Importantly,
% it does not require that anything is known about the catcode situation in
% advance (\emph{cf.}~a more traditional approach to saving the state of
% targeting characters).
%
% \StopEventually{}
%
% \section{Implementation}
%
%    \begin{macrocode}
%<*package>
%    \end{macrocode}
%
%
% Check for functionality using \cs{newluafunction} as a marker.
%    \begin{macrocode}
\ifx\newluafunction\@undefined
  \ifx\documentclass\@undefined
    
\edef\etatcatcode{\the\catcode`\@}
\catcode`\@=11

\ifx\e@alloc\@undefined\else
\expandafter\endinput
\fi

%
% fixes to etex.src, 
% These could and probably should be made directly in an
% update to etex.src which already has some luatex-specific
% code, but does not define the correct range for luatex.

% 2015-07-13 higher range in luatex
\edef \et@xmaxregs {\ifx\directlua\@undefined 32768\else 65536\fi}
% luatex/xetex also allow more math fam
\edef \et@xmaxfam {\ifx\Umathchar\@undefined\sixt@@n\else\@cclvi\fi}

\count 270=\et@xmaxregs % locally allocates \count registers 32767, 32766, ...
\count 271=\et@xmaxregs % ditto for \dimen registers
\count 272=\et@xmaxregs % ditto for \skip registers
\count 273=\et@xmaxregs % ditto for \muskip registers
\count 274=\et@xmaxregs % ditto for \box registers
\count 275=\et@xmaxregs % ditto for \toks registers
\count 276=\et@xmaxregs % ditto for \marks classes

% and 256 or 16 fam
\outer\def\newfam{\alloc@8\fam\chardef\et@xmaxfam}

% end of proposed changes to etex.src
%%%%%%%%%%%%%%%%%%%%%%%%%%%%%%%%%%%%%%%%%%%%%%%%%%%%%%


% Switch to global cf luatex.sty to leave room for inserts
% not really needed for luatex but possibly most compatible
% with existing use.
\let\newcount\globcount
\let\newdimen\globdimen
\let\newskip\globskip
\let\newbox\globbox

%%%%%%%%%%%%%%%%%%%%%%%%%%%%%%%%%%%%%%%%%%%%%%%%%%%%%%



% define\e@alloc as in latex (the existing macros in etex.src
% hard to extend to further register types as they assume specific
% 26x and 27x count range. For compatibility the existing register
% allocation is not changed.


\def\e@alloc#1#2#3#4#5#6{%
  \global\advance#3\@ne
  \e@ch@ck{#3}{#4}{#5}#1%
  \allocationnumber#3\relax
  \global#2#6\allocationnumber
  \wlog{\string#6=\string#1\the\allocationnumber}}%
\gdef\e@ch@ck#1#2#3#4{%
  \ifnum#1<#2\else
    \ifnum#1=#2\relax
      #1\@cclvi
      \ifx\count#4\advance#1 10 \fi
    \fi
    \ifnum#1<#3\relax
    \else
      \errmessage{No room for a new \string#4}%
    \fi
  \fi}%


%%%%%%
\long\def \@gobble #1{}
\long\def\@firstofone#1{#1}
\def\makeatother{\catcode`\@12\relax}
\input ltluatex.sty\relax


% fix up allocations not to clash with etex.src
% \count registers 256-259 and 267-269 are not (yet) used
% 

% ltluatex uses 258 and 259
% also uses 260 and 261 but change them to 267 and 268
\def\newluafunction{%
  \e@alloc\luafunction\e@alloc@chardef
    {\count267}\m@ne\e@alloc@top
}
\count267=\z@
\def\newwhatsit#1{%
  \e@alloc\whatsit\e@alloc@chardef
    {\count268}\m@ne\e@alloc@top#1%
}
\count268=\z@

\directlua{
local function new_whatsit(name)
  tex_setcount("global", 267, tex_count[267] + 1)
  if tex_count[267] > 65534 then
    latex_error("No room for a new custom whatsit")
    return -1
  end
  texio_write_nl("Custom whatsit " .. name .. " = " .. tex_count[267])
  return tex_count[267]
end
latex.new_whatsit = new_whatsit
}

\catcode`\@=\etatcatcode\relax
%
  \else
    \RequirePackage{ltluatex}
  \fi
\fi
%    \end{macrocode}
%
% \begin{macro}{\@setcatcodetable}
%   A handy utility.
%    \begin{macrocode}1
\def\@setcatcodetable#1#2{%
  \begingroup
    #2%
    \savecatcodetable#1%
  \endgroup
}
%    \end{macrocode}
% \end{macro}
%
% \begin{macro}{\@setrangecatcode}
%   A handy utility
%    \begin{macrocode}
\def\setrangecatcode#1#2#3{%
  \ifnum#1>#2 %
    \expandafter\@gobble
  \else
    \expandafter\@firstofone
  \fi
    {%
      \catcode#1=#3 %
      \expandafter\setrangecatcode\expandafter
        {\number\numexpr#1+1\relax}{#2}{#3}%
    }%
}
%    \end{macrocode}
% \end{macro}
%
% \begin{macro}{\@catcodetablelist}
% \begin{macro}{\@catcodetablestack}
%   Data structures for a stack: a list of free tables in the stack and
%   the stack record itself.
%    \begin{macrocode}
\def\@catcodetablelist{}
\def\@catcodetablestack{}
%    \end{macrocode}
% \end{macro}
% \end{macro}
%
% \begin{macro}{\@catcodetablestackcnt}
%   A count for adding to the list of scratch tables.
%    \begin{macrocode}
\newcount\@catcodetablestackcnt
%    \end{macrocode}
% \end{macro}
%
% \begin{macro}{\@pushcatcodetable}
% \begin{macro}{\@pushctbl}
%   To push a table, first check there is a free one in the pool and if
%   not create one. Then take the top table in the pool and use it to save
%   the current table.
%    \begin{macrocode}
\def\@pushcatcodetable{%
  \ifx\@catcodetablelist\empty
    \global\advance\@catcodetablestackcnt by\@ne
    \edef\@tempa{\csname ct@\the\@catcodetablestackcnt\endcsname}%
    \expandafter\newcatcodetable\@tempa
    \xdef\@catcodetablelist{\@tempa}%
  \fi
  \expandafter\@pushctbl\@catcodetablelist\@nil
}
\def\@pushctbl#1#2\@nil{%
  \gdef\@catcodetablelist{#2}%
  \xdef\@catcodetablestack{#1\@catcodetablestack}%
  \savecatcodetable#1%
}
%    \end{macrocode}
% \end{macro}
% \end{macro}
%
% \begin{macro}{\@popcatcodetable}
% \begin{macro}{\@popctbl}
%   Much the same in reverse.
%    \begin{macrocode}
\def\@popcatcodetable{%
  \ifx\@catcodetablestack\@empty
    \errmessage{Attempt to pop empty catcodetable stack}%
  \fi
  \expandafter\@popctbl\@catcodetablestack\@nil
}
\def\@popctbl#1#2\@nil{%
  \gdef\@catcodetablestack{#2}%
  \xdef\@catcodetablelist{\@catcodetablelist#1}%
  \catcodetable#1%
}
%    \end{macrocode}
% \end{macro}
% \end{macro}
%
%    \begin{macrocode}
%</package>
%    \end{macrocode}
%
% \Finale
