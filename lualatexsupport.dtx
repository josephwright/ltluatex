% \iffalse meta-comment
%
% Copyright (C) 2015 David Carlisle and Joseph Wright
%
% It may be distributed and/or modified under the conditions of
% the LaTeX Project Public License (LPPL), either version 1.3c of
% this license or (at your option) any later version.  The latest
% version of this license is in the file:
%
%   http://www.latex-project.org/lppl.txt
%
% \fi
% \iffalse
%<*package>
\ProvidesPackage{lualatexsupport}
%</package>
%<*!lua>
% \fi
% \ProvidesFile{lualatexsupport.dtx}
  [0000/00/00 v1.0 LuaTeX support add-ons for modern LaTeX2e kernels]
% \iffalse
%</!lua>
%<*driver>
\documentclass{ltxdoc}
\usepackage{lualatexsupport}
\begin{document}
\DocInput{\jobname.dtx}
\end{document}
%</driver>
% \fi
%
% \GetFileInfo{lualatexsupport.dtx}
%
% \title{\filename\\(Lua\TeX{}-specific support)\thanks{This file has
%   version number \fileversion, last revised \filedate.}}
% \author{David Carlisle and Joseph Wright\footnote{Portions
%  of the code here are adapted/simplified from the packages \textsf{luatex}
%  and \textsf{luatexbase} written by Heiko Oberdiek, \'{E}lie Roux,
%  Manuel P\'{e}gouri\'{e}-Gonnar and Philipp Gesang.}}
% \date{\filedate}
%
% \maketitle
%
% \CheckSum{0}
%
% \section{Accessing \TeX{} register numbers from Lua}
%
% \DescribeMacro{latex.registernumber}
% |latex.registernumer(|\meta{name}|)|\\
% Sometimes (notably in the case of Lua attributes) it is necessary to
% access a register \emph{by number} that has been allocated by \TeX{}.
% This package provides a function to look up the relevant number
% using Lua\TeX{}'s internal tables. So after for example
% |\newattribute\myattrib|, |\myattrib| would be defined by (say)
% |\myattrib=\attribute15|. The funtion call |latex.regisernumer("myattrib")|
% will then return the register number, 15 in this case. If the string passed
% as argument does not correspond to a token defined by |\attributedef|,
% |\countdef| or similar commands, the lua value |false| is returned.
%
% \subsection{Example}
%
%\begin{verbatim}
% \newcommand\test[1]{%
% \typeout{#1: \expandafter\meaning\csname#1\endcsname^^J
% \space\space\space\space
% \directlua{tex.write(latex.registernumber("#1") or "bad input")}%
% }}
%
% \test{undefinedrubbish}
%
% \test{space}
%
% \test{hbox}
%
% \test{@MM}
%
% \test{@tempdima}
% \test{@tempdimb}
%
% \test{strutbox}
%
% \test{sixt@@n}
%
% \attrbutedef\myattr=12
% \myattr=200
% \test{myattr}
%
%\end{verbatim}
%
%\ifx\directlua\undefined\else
% \newcommand\test[1]{%
% \typeout{#1: \expandafter\meaning\csname#1\endcsname^^J
% \space\space\space\space
% \directlua{tex.write(latex.registernumber("#1") or "bad input")}%
% }}
%
% \test{undefinedrubbish}
%
% \test{space}
%
% \test{hbox}
%
% \test{@MM}
%
% \test{@tempdima}
% \test{@tempdimb}
%
% \test{strutbox}
%
% \test{sixt@@n}
%
% \luatexattributedef\myattr=12
% \myattr=200
% \test{myattr}
%
%\fi
%
% If this document has been processed with lualatex then the above
% code would have produced the following in the log and terminal
% output.
%\begin{verbatim}
% undefinedrubbish: \relax
%      bad input
% space: macro:->
%      bad input
% hbox: \hbox
%      bad input
% @MM: \mathchar"4E20
%      20000
% @tempdima: \dimen14
%      14
% @tempdimb: \dimen15
%      15
% strutbox: \char"B
%      11
% sixt@@n: \char"10
%      16
% myattr: \attribute12
%      12
%\end{verbatim}
%
% Notice how undefined commands, or commands unreleated to registers
% do not produce an error, just return |false| and so print
% |bad input| here. Note also that commands defined by |\newbox| work and
% return the number of the box register even though the actual command
% holding this number is a |\chardef| defined token (there is no
% |\boxdef|).
%
% \section{Attribute setting}
%
% \DescribeMacro{\setattribute}
% \DescribeMacro{\unsetattribute}
% |\setattribute{|\meta{attribute}|}{|\meta{value}|}|\\
% |\unsetattribute{|\meta{attribute}|}|\\
% Set and unset attributes in a manner analogous to |\setlength|. Note that
% attributes take a marker value when unset so this operation is distinct
% from setting the value to zero.
%
% \section{Category code table support}
%
% \DescribeMacro{\setcatcodetable}
% |\setcatcodetable{|\meta{table}|}{|\meta{catcodes}|}|\\
% Sets the \meta{table} (which must have been previously defined) to
% apply the \meta{catcodes} specified.
%
% \DescribeMacro{\setcatcoderange}
% |\setcatcoderange{|\meta{start}|}{|\meta{end}|}{|\meta{catcode}|}|\\
% Sets all characters in the range \meta{start}--\meta{end} inclusive to
% have the \meta{catcode} specified.
%
% \DescribeMacro{\pushcatcodes}
% \DescribeMacro{\popcatcodes}
% |\pushcatcodes|\\
% |\popcatcodes|\\
% This pair of commands enable the current category code r\'{e}gime to
% be saved and restored meaning that arbitrary catcode changes can be made.
% This functionality will normally be used in concert with applying
% catcode tables. For example
% \begin{verbatim}
% \catcode`\Z=4 %
% \pushcatcodes
% \expandafter\catcodetable\csname catcodetable@latex\endcsname
% % Code here
% \popcatcodes
% \showthe\catcode`\Z
% \end{verbatim}
% will ensure that the `content' is set with normal category codes but
% allow restoration of the non-standard codes at the conclusion. Importantly,
% it does not require that anything is known about the catcode situation in
% advance (\emph{cf.}~a more traditional approach to saving the state of
% targetting characters).
%
% \StopEventually{}
%
% \section{Preliminaries}
%
%    \begin{macrocode}
%<*package>
%    \end{macrocode}
%
% Obviously we need Lua\TeX{} so simply stop if that is not in use.
%    \begin{macrocode}
\ifx\directlua\@undefined
  \PackageInfo{luatexsupport}
    {LuaTeX not in use: package loading aborted.}
  \expandafter\endinput
\fi
%    \end{macrocode}
% Next, check that both the kernel a new enough.
%    \begin{macrocode}
\ifx\newluafunction\@undefined
  \PackageInfo{luatexsupport}
    {Kernel support unavailable: kernel too old.}
  \expandafter\endinput
\fi
\ifnum\luatexversion<60 %
  \PackageInfo{luatexsupport}
    {Kernel support unavailable: LuaTeX too old.}
  \expandafter\endinput
\fi
%    \end{macrocode}
%
% \section{Attributes}
%
% \begin{macro}{\setattribute}
% \begin{macro}{\unsetattribute}
%   Handy utilities.
%    \begin{macrocode}
\newcommand*\setattribute#1#2{#1=\numexpr#2\relax}
\newcommand*\unsetattribute#1{#1=-"7FFFFFFF\relax}
%    \end{macrocode}
% \end{macro}
% \end{macro}
%
% \section{Category code tables}
%
% \begin{macro}{\setcatcodetable}
% \begin{macro}{\setcatcoderange}
%   Handy utilities.
%    \begin{macrocode}
\newcommand*\setcatcodetable[2]{%
  \begingroup
    #2%
    \savecatcodetable#1%
  \endgroup
}
\newcommand*\setcatcoderange[3]{%
  \ifnum#1>#2 %
    \expandafter\@gobble
  \else
    \expandafter\@firstofone
  \fi
    {%
      \catcode#1=#3 %
      \expandafter\setcatcoderange\expandafter
        {\number\numexpr#1+1\relax}{#2}{#3}%
    }%
}
%    \end{macrocode}
% \end{macro}
% \end{macro}
%
% \begin{macro}{\@catcodetablelist}
% \begin{macro}{\@catcodetablestack}
%   Data structures for a stack: a list of free tables in the stack and
%   the stack record itself.
%    \begin{macrocode}
\newcommand*\@catcodetablelist{}
\newcommand*\@catcodetablestack{}
%    \end{macrocode}
% \end{macro}
% \end{macro}
%
% \begin{macro}{\@catcodetablestackcnt}
%   A count for adding to the list of scratch tables.
%    \begin{macrocode}
\newcount\@catcodetablestackcnt
%    \end{macrocode}
% \end{macro}
%
% \begin{macro}{\pushcatcodes}
% \begin{macro}{\@pushctbl}
%   To push a table, first check there is a free one in the pool and if
%   not create one. Then take the top table in the pool and use it to save
%   the current table.
%    \begin{macrocode}
\newcommand*\pushcatcodes{%
  \ifx\@catcodetablelist\@empty
    \global\advance\@catcodetablestackcnt by\@ne
    \edef\@tempa{\@nameuse{ct@\the\@catcodetablestackcnt}}%
    \expandafter\newcatcodetable\@tempa
    \xdef\@catcodetablelist{\@tempa}%
  \fi
  \expandafter\@pushctbl\@catcodetablelist\@nil
}
\def\@pushctbl#1#2\@nil{%
  \gdef\@catcodetablelist{#2}%
  \xdef\@catcodetablestack{#1\@catcodetablestack}%
  \savecatcodetable#1%
}
%    \end{macrocode}
% \end{macro}
% \end{macro}
%
% \begin{macro}{\popcatcodes}
% \begin{macro}{\@popctbl}
%   Much the same in reverse.
%    \begin{macrocode}
\newcommand*\popcurrentcatcodes{%
  \ifx\@catcodetablestack\@empty
    \PackageError{lualatexsupport}
      {Attempt to pop empty catcodetable stack}\@ehc
  \fi
  \expandafter\@popctbl\@catcodetablestack\@nil
}
\def\@popctbl#1#2\@nil{%
  \gdef\@catcodetablestack{#2}%
  \xdef\@catcodetablelist{\@catcodetablelist#1}%
  \catcodetable#1%
}
%    \end{macrocode}
% \end{macro}
% \end{macro}
%
% \section{Accessing register numbers from Lua}
%
% To convert the \TeX{} names into register numbers in Lua we use the
% hash table and the fact that for each type they are stored sequentially.
% That then requires knowing the name of a \TeX{} definition for each
% zero-numbered register. We could use some from the kernel, but within
% a group we can make things easier to read. As a result of these
% requirements, the Lua code is all loaded here.
%    \begin{macrocode}
\begingroup
  \attributedef\attributezero=0 %
  \chardef     \charzero     =0 %
  \countdef    \countzero    =0 %
  \dimendef    \dimenzero    =0 %
  \mathchardef \mathcharzero =0 %
  \muskipdef   \muskipzero   =0 %
  \skipdef     \skipzero     =0 %
  \toksdef     \tokszero     =0 %
  \directlua{require("lualatexsupport")}
\endgroup
%    \end{macrocode}
%
%    \begin{macrocode}
%</package>
%    \end{macrocode}
%
%    \begin{macrocode}
%<*lua>
%    \end{macrocode}
%
% As this code supplements the kernel we use the same table as it
% does: allowed as this is a product written by the same authors for
% a very specific reason!
%    \begin{macrocode}
latex       = latex or { }
local latex = latex
%    \end{macrocode}
%
% Collect up the data from the \TeX{} level into a Lua table: from
% version~0.80, Lua\TeX{} makes that easy.
%    \begin{macrocode}
local luaregisterbasetable = { }
local registermap = {
  attributezero = "assign_attr"    ,
  charzero      = "char_given"     ,
  countzero     = "assign_int"     ,
  dimenzero     = "assign_dimen"   ,
  mathcharzero  = "math_given"     ,
  muskipzero    = "assign_mu_skip" ,
  skipzero      = "assign_skip"    ,
  tokszero      = "assign_toks"    ,
}
local i, j
for i,j in pairs (registermap) do
  if tex.luatexversion < 80 then
    luaregisterbasetable[tex.hashtokens()[i][1]] =
      tex.hashtokens()[i][2]
  else
    luaregisterbasetable[j] = newtoken.create(i).mode
  end
end
%    \end{macrocode}
%
% \begin{macro}{latex.registernumber}
% Working out the correct return value can be done in two ways. For older
% Lua\TeX{} releases it has to be extracted from the |hashtokens|. On the
% otehr hand, newer Lua\TeX{}'s have |newtoken|, and whilst |.mode| isn't
% currentld documented, Hans Hagen pointed to this approach so we should be
% OK.
%    \begin{macrocode}
local registernumber
if tex.luatexversion < 80 then
  function registernumber(name)
    local nt = tex.hashtokens()[name]
    if(nt and luaregisterbasetable[nt[1]]) then
      return nt[2] - luaregisterbasetable[nt[1]]
    else
      return false
    end
  end
else
  function registernumber(name)
    local nt = newtoken.create(name)
    if(luaregisterbasetable[nt.cmdname]) then
      return nt.mode - luaregisterbasetable[nt.cmdname]
    else
      return false
    end
  end
end
latex.registernumber = registernumber
%    \end{macrocode}
% \end{macro}
%
%    \begin{macrocode}
%</lua>
%    \end{macrocode}
%
% \Finale